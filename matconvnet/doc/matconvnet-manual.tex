\documentclass[12pt]{memoir}
\usepackage{amssymb}
\usepackage{amsmath}
\usepackage{url}
\usepackage{xspace}
\usepackage[margin=2.5cm]{geometry}
\usepackage{tikz}
\usepackage{pgfplots} 
\usepackage{listings}
\usepackage{color}
\usepackage{textcomp}
\usepackage{fancyvrb}
\usepackage[colorlinks]{hyperref}

\newcommand{\real}{\mathbb{R}}
\newcommand{\vv}{\operatorname{vec}}
\newcommand{\diag}{\operatorname{diag}}
\newcommand{\vlnn}{\textsc{MatConvNet}\xspace}
\newcommand{\cpp}{C{}\texttt{++}~}
\newcommand{\bx}{\mathbf{x}}
\newcommand{\by}{\mathbf{y}}
\newcommand{\bz}{\mathbf{z}}
\newcommand{\bff}{\mathbf{f}}
\newcommand{\br}{\mathbf{r}}
\newcommand{\bw}{\mathbf{w}}
\newcommand{\bfs}{\mathbf{s}}
\newcommand{\bfe}{\mathbf{e}}
\newcommand{\bone}{\mathbf{1}}
\newcommand{\argmin}{\operatornamewithlimits{argmin}}
\newcommand{\argmax}{\operatornamewithlimits{argmax}}
\newcommand{\sign}{\operatornamewithlimits{sign}}

\tikzstyle{block} = [draw, rectangle, minimum height=3em, minimum width=3em] \tikzstyle{data} = []
\tikzstyle{pinstyle} = [pin edge={to-,thin,black}]

\VerbatimFootnotes

\setsecnumdepth{subsection}
\settocdepth{subsection}

\definecolor{listinggray}{gray}{0.9}
\definecolor{lbcolor}{rgb}{0.8,0.8,0.8}
\lstset{
	%backgroundcolor=\color{lbcolor},
	tabsize=4,
	rulecolor=,
	language=matlab,
	basicstyle=\footnotesize,
	upquote=true,
	columns=fixed,
	aboveskip={1.2\baselineskip},
	belowskip={1.2\baselineskip},
	showstringspaces=false,
	extendedchars=true,
	breaklines=false,
	prebreak = \raisebox{0ex}[0ex][0ex]{\ensuremath{\hookleftarrow}},
	%frame=single,
	showtabs=false,
	showspaces=false,
	showstringspaces=false,
	identifierstyle=\ttfamily,
	keywordstyle=\color[rgb]{0,0,1},
	commentstyle=\itshape\color[rgb]{0.133,0.545,0.133},
	stringstyle=\color[rgb]{0.627,0.126,0.941},
}

\title{MatConvNet \\
\Large
Convolutional Neural Networks for MATLAB}
\author{
Andrea Vedaldi
\and
Karel Lenc}
\date{}

% ------------------------------------------------------------------
\begin{document}
% ------------------------------------------------------------------

\frontmatter
\maketitle{}
\clearpage

\begin{abstract}
\vlnn is an implementation of Convolutional Neural Networks (CNNs) for MATLAB. The toolbox is designed with an emphasis on simplicity and flexibility. It exposes the building blocks of CNNs as easy-to-use MATLAB functions, providing routines for computing linear convolutions with filter banks, feature pooling, and many more. In this manner, \vlnn allows fast prototyping of new CNN architectures; at the same time, it supports efficient computation on CPU and GPU allowing to train complex models on large datasets such as ImageNet ILSVRC. This document provides an overview of CNNs and how they are implemented in \vlnn and gives the technical details of each computational block in the toolbox.
\end{abstract}
\clearpage

\tableofcontents*
\clearpage

\mainmatter
% ------------------------------------------------------------------
\chapter{Introduction to MatConvNet}\label{s:intro}
% ------------------------------------------------------------------

\vlnn is a simple MATLAB toolbox implementing Convolutional Neural Networks (CNN) for computer vision applications. This documents starts with a short overview of CNNs and how they are implemented in \vlnn. Section~\ref{s:blocks} lists all the computational building blocks implemented in \vlnn that can be combined to create CNNs and gives the technical details of each one. Finally, Section~\ref{s:wrappers} discusses more abstract CNN wrappers and example code and models.

A \emph{Convolutional Neural Network} (CNN) can be viewed as a function $f$ mapping data $\bx$, for example an image, on an output vector $\by$. The function $f$ is a composition of a sequence (or a directed acyclic graph) of simpler functions $f_1,\dots,f_L$, also called \emph{computational blocks} in this document. Furthermore, these blocks are \emph{convolutional}, in the sense that they map an input image of feature map to an output feature map by applying a translation-invariant and local operator, e.g. a linear filter. The \vlnn toolbox contains implementation for the most commonly used computational blocks (described in Section~\ref{s:blocks}) which can be used either directly, or through simple wrappers. Thanks to the modular structure, it is a simple task to create and combine new blocks with the existing ones. %New blocks are also easy to create and combine with the existing ones.

Blocks in the CNN usually contain parameters $\bw_1,\dots,\bw_L$. These are \emph{discriminatively learned from example data} such that the resulting function $f$ realizes an useful mapping. A typical example is image classification; in this case the output of the CNN is a vector $\by=f(\bx)\in\real^C$ containing the confidence that $\bx$ belong to any of $C$ possible classes. Given training data $(\bx^{(i)},\by^{(i)})$ (where $\by^{(i)}$ is the indicator vector of the class of $\bx^{(i)}$), the parameters are learned by solving
\begin{equation}\label{e:objective}
 \argmin_{\bw_1,\dots\bw_n}
 \frac{1}{n}\sum_{i=1}^n
 \ell\left(
 f(\bx^{(i)};\bw_1,\dots,\bw_L),
 \by^{(i)}
 \right)
\end{equation}
where $\ell$ is a suitable \emph{loss function} (e.g. the hinge or log loss).

The optimization problem~\eqref{e:objective} is usually non-convex and very large as complex CNN architectures need to be trained from hundred-thousands or even millions of examples. Therefore efficiency is a paramount. Optimization often uses a variant of \emph{stochastic gradient descent}. The algorithm is, conceptually, very simple: at each iteration a training point is selected at random, the derivative of the loss term for that training sample is computed resulting in a gradient vector, and parameters are incrementally updated by moving towards the local minima in the direction of the gradient. The key operation here is to compute the derivative of the objective function, which is obtained by an application of the chain rule known as \emph{back-propagation}. \vlnn can evaluate the derivatives of all the computational blocks. It also contains several examples of training small and large models using these capabilities and a default solver, although it is easy to write customized solvers on top of the library.

While CNNs are relatively efficient to compute, training requires iterating many times through vast data collections. Therefore the computation speed is very important in practice. Larger models, in particular, may require the use of GPU to be trained in a reasonable time. \vlnn has integrated GPU support based on NVIDIA CUDA and MATLAB built-in CUDA capabilities.

% ------------------------------------------------------------------
\section{\vlnn at a glance}\label{s:vlnn}
% ------------------------------------------------------------------

\vlnn has a simple design philosophy. Rather than wrapping CNNs around complex layers of software, it exposes simple functions to compute CNN building blocks, such as linear convolution and ReLU operators. These building blocks are easy to combine into a complete CNNs and can be used to implement sophisticated learning algorithms. While several real-world examples of small and large CNN architectures and training routines are provided, it is always possible to go back to the basics and build your own, using the efficiency of MATLAB in prototyping. Often no C coding is required at all to try a new architectures. As such, \vlnn is an ideal playground for research in computer vision and CNNs.

\vlnn contains the following elements:
\begin{itemize}
\item \emph{CNN computational blocks.} A set of optimized routines computing fundamental building blocks of a CNN. For example, a convolution block is implemented by \linebreak \verb!y=vl_nnconv(x,f,b)! where \verb!x! is an image, \verb!f! a filter bank, and \verb!b! a vector of biases (Section~\ref{s:convolution}). The derivatives are computed as
\verb![dzdx,dzdf,dzdb] = vl_nnconv(x,f,b,dzdy)! where \verb!dzdy! is the derivative of the CNN output w.r.t \verb!y!~(Section~\ref{s:convolution}). Section~\ref{s:blocks} describes all the blocks in detail.
\item \emph{CNN wrappers.} \vlnn provides a simple wrapper, suitably invoked by \verb!vl_simplenn!, that implements a CNN with a linear topology (a chain of blocks). This is good enough to run most of current state-of-the-art models for image classification. You are invited to look at the implementation of this function, as it is a great starting point to understand how to implement more complex CNNs.
\item \emph{Example applications.} \vlnn provides several example of learning CNNs with stochastic gradient descent and CPU or GPU, on MNIST, CIFAR10, and ImageNet data.
\item \emph{Pre-trained models.} \vlnn provides several state-of-the-art pre-trained CNN models that can be used off-the-shelf, either to classify images or to produce image encodings in the spirit of Caffe or DeCAF.
\end{itemize}

% ------------------------------------------------------------------
\subsection{The structure and evaluation of CNNs}\label{s:forward}
% ------------------------------------------------------------------

CNNs are obtained by connecting one or more \emph{computational blocks}. Each block $\by = f(\bx,\bw)$ takes an image $\bx$ and a set of parameters $\bw$ as input and produces a new image $\by$ as output. An image is a real 4D array; the first two dimensions index spatial coordinates (image rows and columns respectively), the third dimension feature channels (there can be any number), and the last dimension image instances. A computational block $f$ is therefore represented as follows:
\begin{center}
\begin{tikzpicture}[auto, node distance=2cm]
\node (x) [data] {$\bx$};
\node (f) [block,right of=x]{$f$};
\node (y) [data, right of=f] {$\by$};
\node (w) [data, below of=f] {$\bw$};
\draw [->] (x.east) -- (f.west) {};
\draw [->] (f.east) -- (y.west) {};
\draw [->] (w.north) -- (f.south) {};
\end{tikzpicture}
\end{center}
Formally, $\bx$ is a 4D tensor stacking $N$ 3D images
\[
   \bx \in \real^{H \times W \times D \times N}
\]
where $H$ and $W$ are the height and width of the images, $D$ its depth, and $N$ the number of images. In what follows, all operations are applied identically to each image in the stack $\bx$; hence for simplicity we will drop the last dimension in the discussion (equivalent to assuming $N=1$), but the ability to operate on image batches is very important for efficiency.

In general, a CNN can be obtained by connecting blocks in a directed acyclic graph (DAG). In the simplest case, this graph reduces to a sequence of computational blocks $(f_1,f_2,\dots,f_L)$. Let $\bx_1,\bx_2,\dots,\bx_L$ be the output of each layer in the network, and let $\bx_0$ denote the network input. Each output $\bx_l$ depends on the previous output $\bx_{l-1}$ through a function $f_l$ with parameter $\bw_l$ as $\bx_l = f_l(\bx_{l-1};\bw_l)$; schematically:
\begin{center}
\begin{tikzpicture}[auto, node distance=2cm]
\node (x0)  [data] {$\bx_0$};
\node (f1) [block,right of=x0]{$f_1$};
\node (f2) [block,right of=f1,node distance=3cm]{$f_2$};
\node (dots) [right of=f2]{...};
\node (fL) [block,right of=dots]{$f_L$};
\node (xL)  [data, right of=fL] {$\bx_L$};
\node (w1) [data, below of=f1] {$\bw_1$};
\node (w2) [data, below of=f2] {$\bw_2$};
\node (wL) [data, below of=fL] {$\bw_L$};
\draw [->] (x0.east) -- (f1.west) {};
\draw [->] (f1.east) -- node {$\bx_2$} (f2.west);
\draw [->] (f2.east) -- node {$\bx_3$} (dots.west) {};
\draw [->] (dots.east) -- node {$\bx_{L-1}$} (fL.west) {};
\draw [->] (fL.east) -- (xL.west) {};
\draw [->] (w1.north) -- (f1.south) {};
\draw [->] (w2.north) -- (f2.south) {};
\draw [->] (wL.north) -- (fL.south) {};
\end{tikzpicture}
\end{center}
Given an input $\bx_0$, evaluating the network is a simple matter of evaluating all the intermediate stages in order to compute an overall function $\bx_L = f(\bx_0;\bw_1,\dots,\bw_L)$. 

% ------------------------------------------------------------------
\subsection{CNN derivatives}\label{s:backward}
% ------------------------------------------------------------------

In training a CNN, we are often interested in taking the derivative of a loss $\ell : f(\bx,\bw) \mapsto \real$ with respect to the parameters. This effectively amounts to extending the network with a \emph{scalar block} at the end:
\begin{center}
\begin{tikzpicture}[auto, node distance=2cm]
\node (x0)  [data] {$\bx_0$};
\node (f1) [block,right of=x0]{$f_1$};
\node (f2) [block,right of=f1,node distance=3cm]{$f_2$};
\node (dots) [right of=f2]{...};
\node (fL) [block,right of=dots]{$f_L$};
\node (loss) [block,right of=fL,node distance=3cm]{$\ell$};
\node (w1) [data, below of=f1] {$\bw_1$};
\node (w2) [data, below of=f2] {$\bw_2$};
\node (wL) [data, below of=fL] {$\bw_L$};
\node (z) [data, right of=loss] {$z\in\real$};
\draw [->] (x0.east) -- (f1.west) {};
\draw [->] (f1.east) -- node {$\bx_2$} (f2.west);
\draw [->] (f2.east) -- node {$\bx_3$} (dots.west) {};
\draw [->] (dots.east) -- node {$\bx_{L-1}$} (fL.west) {};
\draw [->] (fL.east) -- node {$\bx_L$} (loss.west);
\draw [->] (loss.east) -- (z) {};
\draw [->] (w1.north) -- (f1.south) {};
\draw [->] (w2.north) -- (f2.south) {};
\draw [->] (wL.north) -- (fL.south) {};
\end{tikzpicture}
\end{center}
The derivative of $\ell \circ f$ with respect to the parameters can be computed but starting from the end of the chain (or DAG) and working backwards using the chain rule, a process also known as back-propagation. For example the derivative w.r.t. $\bw_l$ is:
\begin{equation}\label{e:chain-rule}
 \frac{dz}{d(\vv\bw_l)^\top}
 =
 \frac{dz}{d(\vv\bx_{L})^\top}
 \frac{d\vv\bx_{L}}{d(\vv\bx_{L-1})^\top}
 \dots
 \frac{d\vv\bx_{l+1}}{d(\vv\bx_{l})^\top}
 \frac{d\vv\bx_{l}}{d(\vv\bw_{l})^\top}.
\end{equation}
Note that the derivatives are implicitly evaluated at the working point determined by the input $\bx_0$ during the evaluation of the network in the forward pass. The $\vv$ symbol is the vectorization operator, which simply reshape its tensor argument to a column vector. This notation for the derivatives is taken from~\cite{kinghorn96integrals} and is used throughout this document.

Computing~\eqref{e:chain-rule} requires computing the derivative of each block $\bx_l = f_l(\bx_{l-1},\bw_l)$ with respect to its parameters $\bw_l$ and input $\bx_{l-1}$. Let us know focus on computing the derivatives for one computational block. We can look at the network as follows:
\[
    \underbrace{
    \ell \circ f_{L}(\cdot,\bw_L)
     \circ f_{L-1}(\cdot,\bw_{L-1})
     \dots
     \circ f_{l+1}(\cdot,\bw_{l+1})
     }_{\displaystyle z(\cdot)}
     \circ f_{l}(\bx_l,\bw_{l})
     \circ \dots
\]
where $\circ$ denotes the composition of function. For simplicity, lump together the factors from $f_l+1$ to the loss $\ell$ into a single scalar function $z(\cdot)$ and drop the subscript $l$ from the first block. Hence, the problem is to compute the derivative of $(z \circ f)(\bx,\bw) \in \real$ with respect to the data $\bx$ and the parameters $\bw$. Graphically:
\begin{center}
\begin{tikzpicture}[auto, node distance=2cm]
\node (x) [data] {$\bx$};
\node (f) [block,right of=x ] {$f$};
\node (bz)[block,right of=f ] {$z(\cdot)$};
\node (z) [data, right of=bz] {$z$};
\node (w) [data, below of=f ] {$\bw$};
\draw [->] (x.east) -- (f.west) {};
\draw [->] (f.east) -- node {$\by$}  (bz.west) {};
\draw [->] (w.north) -- (f.south) {};
\draw [->] (bz.east) -- (z.west) {};
\end{tikzpicture}
\end{center}
The derivative of $z \circ f$ with respect to $\bx$ and $\bw$ are given by:
\[
\frac{dz}{d(\vv \bx)^\top}
=
\frac{dz}{d(\vv \by)^\top}
\frac{d\vv f}{d(\vv \bx)^\top},
\quad
\frac{dz}{d(\vv \bw)^\top}
=
\frac{dz}{d(\vv \by)^\top}
\frac{d\vv f}{d(\vv \bw)^\top},
\]
We note two facts. The first one is that, since $z$ is a scalar function, the derivatives have a number of elements equal to the number of parameters. So in particular $dz/d\vv \bx^\top$ can be reshaped into an array $dz/d\bx$ with the same shape of $\bx$, and the same applies to the `derivatives $dz/d\by$ and $dz/d\bw$. Beyond the notational convenience, this means that storage for the derivatives is not larger than the storage required for the model parameters and forward evaluation.

The second fact is that computing $dz/d\bx$ and $dz/d\bw$ requires the derivative $dz/d\by$. The latter can be obtained by applying this calculation recursively to the next block in the chain.

% ------------------------------------------------------------------
\subsection{CNN modularity}\label{s:modularity}
% ------------------------------------------------------------------

Sections~\ref{s:forward} and~\ref{s:backward} suggests a modular programming interface for the implementation of CNN modules. Abstractly, we need two functionalities:
\begin{itemize}
\item \emph{Forward messages:} Evaluation of the output $\by=f(\bx,\bw)$ given input data $\bx$ and parameters $\bw$ (forward message).
\item \emph{Backward messages:} Evaluation of the CNN derivative $dz/d\bx$ and $dz/d\bw$ with respect to the block input data $\bx$ and parameters $\bw$ given the block input data $\bx$ and paramters $\bw$ as well as the CNN derivative $dx/d\by$ with respect to the block output data $\by$.
\end{itemize}

% ------------------------------------------------------------------
\subsection{Working with DAGs}\label{s:dag}
% ------------------------------------------------------------------

CNN can also be obtained from more complex composition of functions forming a DAG. There are $n+1$ variables $\bx_i$ and $n$ functions $f_i$ with corresponding arguments $\br_{ik}$:
\begin{align*}
\bx_0 &\\
\bx_1 &= f_1(\br_{1,1},\dots,\br_{1,m_1}),\\
\bx_2 &= f_2(\br_{2,1},\dots,\br_{2,m_2}),\\
&\vdots\\
\bx_n &= f_n(\br_{n,1},\dots,\br_{n,m_n})
\end{align*}
Variables are connected to arguments by relations
\[
     \br_{ik} = \bx_{\pi_{ik}}
\]
where $\pi_{ik}$ denotes the variable $\bx_{\pi_{ik}}$  that feeds argument $\br_{ik}$. Together with the implicit fact that each argument $\br_{ik}$ feeds into the  variable $\bx_i$ through function $f_i$, this defines a bipartite DAG representing the dependencies  between variables and argument. The DAG is bipartite. This DAG must be acyclic, hence assuring that, given the value of the input $\bx_0$, all other variables can be evaluated iteratively. Due to this property, without loss of generality we will assume that variables $\bx_0,\bx_1,\dots,\bx_n$ are sorted such that $\bx_i$ depends only on variables that come before in the order (i.e. one always has $\pi_{ik} < i$).

Now assume that $\bx_n = z$ is a scalar network output (usually the learning loss). As before, we are interested in computing derivatives of this function with respect to the network variables. In order to do so, write the output of the DAG $z(\bx_i)$ as a function of the variable $\bx_i$; this has the following interpretation: the DAG is modified by removing function $f_i$ and making $\bx_i$ an input, setting $\bx_i$ to the specified value, setting all the other inputs to the working point at which the derivative should be computed, and evaluating the resulting DAG. Likewise, one can define functions $z(\br_{ij})$ for the arguments. The derivative with respect to $\bx_j$ can then be expressed as
\[
   \frac{dz}{d(\vv \bx_j)^\top}
   =
  \sum_{(i,k): \pi_{ik}=j}
  \frac{dz}{d(\vv \bx_i)^\top}
  \frac{d\vv \bx_i}{d(\vv \br_{ik})^\top}
  \frac{d\vv \br_{ik}}{d(\vv \bx_j)^\top}
   =
   \sum_{(i,k): \pi_{ik}=j}
   \frac{dz}{d(\vv \bx_i)^\top}
   \frac{d\vv f_i}{d(\vv \br_{ik})^\top}.
\]
Note that these quantities can be computer recursively, backward from the output $z$. In this case, each variable node $\bx_i$ (corresponding to $f_i$) stores the derivative  $dz/d \bx_i$. When node $\bx_i$ is processed, the derivatives $d\vv f_i /d(\vv \br_{ik})^\top$ are computed for the $m_i$ arguments of the corresponding function $f_i$ pre-multiplied by the factor $dz/d \bx_i$ (available at that node) and then accumulated to the corresponding parent variables $\bx_j$'s derivatives $dz/d \bx_j$.

% ------------------------------------------------------------------
\section{About \vlnn}
% ------------------------------------------------------------------

\vlnn main features are:
\begin{itemize}
\item \emph{Flexibility.} Neural network layers are implemented in a straightforward manner, often directly in MATLAB code, so that they are easy to modify, extend, or integrate with new ones.
\item \emph{Power.} The implementation can run the latest models such as Krizhevsky~\textit{et al.}~\cite{krizhevsky12imagenet}, including the DeCAF and Caffe variants, and variants from the Oxford Visual Geometry Group. Pre-learned features for different tasks can be easily downloaded.
\item \emph{Efficiency.} The implementation is quite efficient, supporting both CPU and GPU computation (in the latest versions of MALTAB).
\item \emph{Self contained.} The implementation is fully self-contained, requiring only MATLAB and a compatible C/C++ compiler to work (GPU code requires the freely-available CUDA DevKit). Several fully-functional image classification examples are included.\end{itemize}

\paragraph{Relation to other CNN implementations.} There are many other open-source CNN implementations. \vlnn borrows its convolution algorithms from Caffe (and is in fact capable of running most of Caffe's models). Caffe is a \cpp framework using a custom CNN definition language based on Google Protocol Buffers. Both \vlnn and Caffe are predated by Cuda-Convnet~\cite{krizhevsky12imagenet}, a \cpp-based project that allows defining a CNN architectures using configuration files. While Caffe and Cuda-Convnet can be somewhat faster than \vlnn, the latter exposes individual CNN building blocks as MATLAB functions, as well as integrating with the native MATLAB GPU support, which makes it very convenient for fast prototyping. The DeepLearningToolbox~\cite{deepltbx12} is a MATLAB toolbox implementing, among others, CNNs, but it does not seem to have been tested on large scale problems. While \vlnn specialises on CNNs and computer vision applications, there are several general-purpose machine learning frameworks which include CNN support, but none of them interfaces natively with MATLAB. For example, the Torch7 toolbox~\cite{collobert2011torch7} uses Lua and Theano~\cite{bergstra2010} uses Python.

% ------------------------------------------------------------------
\subsection{Acknowledgments}\label{s:ack}
% ------------------------------------------------------------------

The implementation of several CNN computations in this library are inspired by the Caffe library~\cite{jia13caffe} (however, Caffe is \emph{not} a dependency). Several of the example networks have been trained by Karen Simonyan as part of~\cite{chatfield14return}.

We kindly thank NVIDIA for suppling GPUs used in the creation of this software.


% ------------------------------------------------------------------
\chapter{Network wrappers and examples}\label{s:wrappers}
% ------------------------------------------------------------------

It is easy enough to combine the computational blocks of Sect.~\ref{s:blocks} in any network DAG by writing a corresponding MATLAB script. Nevertheless, \vlnn provides a simple wrapper for the common case of a linear chain. This is implemented by the \verb!vl_simplenn! and \verb!vl_simplenn_move! functions.

\verb!vl_simplenn! takes as input a structure \verb!net! representing the CNN as well as input \verb!x! and potentially output derivatives \verb!dzdy!, depending on the mode of operation. Please refer to the inline help of the \verb!vl_simplenn! function for details on the input and output formats. In fact, the implementation of \verb!vl_simplenn! is a good example of how the basic neural net building block can be used together and can serve as a basis for more complex implementations.

% ------------------------------------------------------------------
\section{Pre-trained models}
% ------------------------------------------------------------------

\verb!vl_simplenn! is easy to use with pre-trained models (see the homepage to download some). For example, the following code downloads a model pre-trained on the ImageNet data and applies it to one of MATLAB stock images:
\begin{lstlisting}[language=Matlab]
% setup MatConvNet in MATLAB
run matlab/vl_setupnn

% download a pre-trained CNN from the web
urlwrite(...
  'http://www.vlfeat.org/sandbox-matconvnet/models/imagenet-vgg-f.mat', ...
  'imagenet-vgg-f.mat') ;
net = load('imagenet-vgg-f.mat') ;

% obtain and preprocess an image
im = imread('peppers.png') ;
im_ = single(im) ; % note: 255 range
im_ = imresize(im_, net.normalization.imageSize(1:2)) ;
im_ = im_ - net.normalization.averageImage ;
\end{lstlisting}
Note that the image should be preprocessed before running the network. While preprocessing specifics depend on the model, the pre-trained model contain a \verb!net.normalization! field that describes the type of preprocessing that is expected. Note in particular that this network takes images of a fixed size as input and requires removing the mean; also, image intensities are normalized in the range [0,255].

The next step is running the CNN. This will return a \verb!res! structure with the output of the network layers:
\begin{lstlisting}[language=Matlab]
% run the CNN
res = vl_simplenn(net, im_) ;
\end{lstlisting}

The output of the last layer can be used to classify the image. The class names are contained in the \verb!net! structure for convenience:
\begin{lstlisting}[language=Matlab]
% show the classification result
scores = squeeze(gather(res(end).x)) ;
[bestScore, best] = max(scores) ;
figure(1) ; clf ; imagesc(im) ;
title(sprintf('%s (%d), score %.3f',...
net.classes.description{best}, best, bestScore)) ;
\end{lstlisting}

Note that several extensions are possible. First, images can be cropped rather than rescaled. Second, multiple crops can be fed to the network and results averaged, usually for improved results. Third, the output of the network can be used as generic features for image encoding.

% ------------------------------------------------------------------
\section{Learning models}
% ------------------------------------------------------------------

As \vlnn can compute derivatives of the CNN using back-propagation, it is simple to implement learning algorithms with it. A basic implementation of stochastic gradient descent is therefore straightforward. Example code is provided in \verb!examples/cnn_train!. This code is flexible enough to allow training on NMINST, CIFAR, ImageNet, and probably many other datasets. Corresponding examples are provided in the \verb!examples/! directory.

% ------------------------------------------------------------------
\section{Running large scale experiments}
% ------------------------------------------------------------------

For large scale experiments, such as learning a network for ImageNet, a NVIDIA GPU (at least 6GB of memory) and adequate CPU and disk speeds are highly recommended. For example, to train on ImageNet, we suggest the following:
\begin{itemize}
\item Download the ImageNet data~\url{http://www.image-net.org/challenges/LSVRC}. Install it somewhere and link to it from \verb!data/imagenet12!
\item Consider preprocessing the data to convert all images to have an height 256 pixels. This can be done with the supplied \verb!utils/preprocess-imagenet.sh! script. In this manner, training will not have to resize the images every time. Do not forget to point the training code to the pre-processed data.
\item Consider copying the dataset in to a RAM disk (provided that you have enough memory!) for faster access. Do not forget to point the training code to this copy.
\item Compile \vlnn with GPU support. See the homepage for instructions.
\end{itemize}

Once your setup is ready, you should be able to run \verb!examples/cnn_imagenet! (edit the file and change any flag as needed to enable GPU support and image pre-fetching on multiple threads).

If all goes well, you should expect to be able to train with 200-300 images/sec.
% ------------------------------------------------------------------
\chapter{Computational blocks}\label{s:blocks}
% ------------------------------------------------------------------

This chapters describes the individual computational blocks supported by \vlnn. The interface of a CNN computational block \verb!<block>! is designed after the discussion in Section~\ref{s:modularity}. The block is implemented as a MATLAB function \verb!y = vl_nn<block>(x,w)! that takes as input MATLAB arrays \verb!x! and \verb!w! representing the input data and parameters and returns an array \verb!y! as output. In general, \verb!x! and \verb!y! are 4D real arrays packing $N$ maps or images, as discussed above, whereas \verb!w! may have an arbitrary shape.

The function implementing each block is capable of working in the backward direction as well, in order to compute derivatives. This is done by passing a third optional argument \verb!dzdy! representing the derivative of the output of the network with respect to $\by$; in this case, the function returns the derivatives \verb![dzdx,dzdw] = vl_nn<block>(x,w,dzdy)! with respect to the input data and parameters. The arrays \verb!dzdx!, \verb!dzdy! and \verb!dzdw! have the same dimensions of \verb!x!, \verb!y! and \verb!w! respectively (see Section~\ref{s:backward}).

Different functions may use a slightly different syntax, as needed: many functions can take additional optional arguments, specified as a property-value pairs; some do not have parameters  \verb!w! (e.g. a rectified linear unit); others can take multiple inputs and parameters, in which case there may be more than one \verb!x!, \verb!w!, \verb!dzdx!, \verb!dzdy! or \verb!dzdw!. See the rest of the chapter and MATLAB inline help for details on the syntax.\footnote{Other parts of the library will wrap these functions into objects with a perfectly uniform interface; however, the low-level functions aim at providing a straightforward and obvious interface even if this means differing slightly from block to block.}

The rest of the chapter describes the blocks implemented in \vlnn, with a particular focus on their analytical definition. Refer instead to MATLAB inline help for further details on the syntax.

% ------------------------------------------------------------------
\section{Convolution}\label{s:convolution}
% ------------------------------------------------------------------

The convolutional block is implemented by the function \verb!vl_nnconv!. \verb!y=vl_nnconv(x,f,b)! computes the convolution of the input map $\bx$ with a bank of $K$ multi-dimensional filters $\bff$ and biases $b$. Here
\[
 \bx\in\real^{H \times W \times D}, \quad
 \bff\in\real^{H' \times W' \times D \times D''}, \quad
 \by\in\real^{H'' \times W'' \times D''}.
\]
Formally, the output  is given by
\[
y_{i''j''d''}
=
b_{d''}
+
\sum_{i'=1}^{H'}
\sum_{j'=1}^{W'}
\sum_{d'=1}^D
f_{i'j'd} \times x_{i''+i'-1,j''+j'-1,d',d''}.
\]
The call \verb!vl_nnconv(x,f,[])! does not use the biases. Note that the function works with arbitrarily sized inputs and filters (as opposed to, for example, square images). See Section~\ref{s:impl-convolution} for technical details.

\paragraph{Padding and stride.} \verb!vl_nnconv! allows to specify  top-bottom-left-right paddings $(P_h^-,P_h^+,P_w^-,P_w^+)$ of the input array and subsampling strides $(S_h,S_w)$ of the output array:
\[
y_{i''j''d''}
=
b_{d''}
+
\sum_{i'=1}^{H'}
\sum_{j'=1}^{W'}
\sum_{d'=1}^D
f_{i'j'd} \times x_{S_h (i''-1)+i'-P_h^-, S_w(j''-1)+j' - P_w^-,d',d''}.
\]
In this expression, the array $\bx$ is implicitly extended with zeros as needed.

\paragraph{Output size.} \verb!vl_nnconv! computes only the ``valid'' part of the convolution; i.e. it requires each application of a filter to be fully contained in the input support. The height of the output array (the width following the same rule) is then
\[
  H'' = H - H' + 1.
\]
When padding is used, the input array is virtually extended with zeros, so the height of the output array is larger:
\[
  H'' = H - H' + 1 + P_h^- + P_h^+.
\]
When downsampling is used as well, this becomes
\[
  H'' = 1 + \left\lfloor \frac{H - H' + P_h^- + P_h^+}{S_h} \right\rfloor.
\]
In general, the padded input must be at least as large as the filters: $H +P_h^- + P_h^+ \geq H'$.

\paragraph{Receptive field size and geometric transformations.} Very often it is useful to relate geometrically the indexes of the various array to the input data (usually images) in term of coordiante transformations and size of the receptive field (i.e. of the image region that affects an output). This is derived later, in Section~\ref{s:receptive}.

\paragraph{Fully connected layers.} In other libraries, a \emph{fully connected blocks or layers} are linear functions where each output dimension depends on all the input dimensions. \vlnn does not distinguishes between fully connected layers and convolutional blocks. Instead, the former is a special case of the latter obtained when the output map $\by$ has dimensions $W''=H''=1$. Internally, \verb!vl_nnconv! handle this case more efficiently when possible.

\paragraph{Filter groups.} For additional flexibility, \verb!vl_nnconv! allows to group channels of the input array $\bx$ and apply to each group different subsets of filters. To use this feature, specify as input a bank  of $D''$ filters $\bff\in\real^{H'\times W'\times D'\times D''}$ such that $D'$ divides the number of input dimensions $D$. These are treated as $g=D/D'$ filter groups; the first group is applied to dimensions $d=1,\dots,D'$ of the input $\bx$; the second group to dimensions $d=D'+1,\dots,2D'$ and so on. Note that the output is still an array $\by\in\real^{H''\times W''\times D''}$.

An application of grouping is implementing the Krizhevsky and Hinton network~\cite{krizhevsky12imagenet} which uses two such streams. Another application is sum pooling; in the latter case, one can specify $D$ groups of $D'=1$ dimensional filters identical filters of value 1 (however, this is considerably slower than calling the dedicated pooling function as given in Section~\ref{s:pooling}).

% ------------------------------------------------------------------
\section{Convolution transpose (deconvolution)}\label{s:convt}
% ------------------------------------------------------------------

The \emph{convolution transpose} block (some authors call it improperly deconvolution) implements an operator this is the transpose of the convolution described in Section~\ref{s:convolution}. This is  implemented by thefunction \verb!vl_nnconvt!. Let:
\[
 \bx\in\real^{H \times W \times D}, \quad
 \bff\in\real^{H' \times W' \times D \times D''}, \quad
 \by\in\real^{H'' \times W'' \times D''}, \quad
\]
be the input tensor, filters, and output tensor. Imagine operating in the reverse direction and to use the filter bank $\bff$ to convolve $\by$ to obtain $\bx$ as in Section~\ref{s:convolution}; since this operation is linear, it can be expressed as a matrix $M$ such that  $\vv \bx = M \vv\by$; convolution transpose computes instead $\vv \by = M^\top \vv \bx$.

Convolution transpose is mainly useful in two settings. The first one are the so called \emph{deconvolutional networks}~\cite{zeiler14visualizing} and other networks such as convolutional decoders that use the transpose of convolution. The second one is implementing data interpolation. In fact, as the convolution block supports input padding and output downsampling, the convolution transpose block supports input upsampling and output cropping. 

Convolution transpose can be expressed in closed form in the following rather unwieldy expression (derived in Section~\ref{s:impl-convolution-transpose}):
\begin{multline}\label{e:convt}
y_{i''j''d''} = 
 \sum_{d'=1}^{D}
 \sum_{i'=0}^{q(H',S_h)}
 \sum_{j'=0}^{q(W',S_w)}
f_{
1+ S_hi' + m(i''+ P_h^-, S_h),\ %
1+ S_wj' + m(j''+ P_w^-, S_w),\ %
d'',
d'
} 
\times \\
x_{
1 - i' + q(i''+P_h^-,S_h),\ %
1 - j' + q(j''+P_w^-,S_w),\ %
d'
}
\end{multline}
where
\[
m(k,S) = (k - 1) \bmod S,
\qquad
q(k,n) = \left\lfloor \frac{k-1}{S} \right\rfloor,
\]
$(S_h,S_w)$ are the vertical and horizontal \emph{input upsampling factors},  $(P_h^-,P_h^+,P_h^-,P_h^+)$ the \emph{output crops}, and $\bx$ and $\bff$ are zero-padded as needed in the calculation. Furthermore, given the input and filter heights, the output height is given by
\[
  H'' = S_h (H - 1) + H' -P^-_h - P^+_h.
\]
A similar formula holds true for the width.

To gain some intuition on this expression, consider only the vertical direction, assume that the crop parameters are zero, and that $S_h>1$. Consider the output sample $y_{i''}$ where the index $i''$ is chose such that $S_h$ divides $i''-1$; according to~\eqref{e:convt}, this sample is obtained as a weighted summation of $x_{i'' / S_h},x_{i''/S_h-1},...$ (note that the order is reversed). The weights are the filter elements $f_1$, $f_{S_h}$,$f_{2S_h},\dots$ subsampled with a step of $S_h$. Now consider computing the element $y_{i''+1}$; due to the rounding in the quotient operation $q(i'',S_h)$, this output sampled is obtained as a weighted combination of the same elements of the input $x$ that were used to compute $y_{i''}$; however, the filter weights are now shifted by one place to the right: $f_2$, $f_{S_h+1}$,$f_{2S_h+1}$, $\dots$. The same is true for $i''+2, i'' + 3,\dots$ until we hit $i'' + S_h$. Here the cycle restarts after shifting $\bx$ to the right by one place. Effectively, convolution transpose works as an \emph{interpolating filter}.

Note also that filter $k$ is stored as a slice $\bff_{:,:,k,:}$ of the 4D tensor $\bff$.

% ------------------------------------------------------------------
\section{Spatial pooling}\label{s:pooling}
% ------------------------------------------------------------------

\verb!vl_nnpool! implements max and sum pooling. The \emph{max pooling} operator computes the maximum response of each feature channel in a $H' \times W'$ patch
\[
y_{i''j''d} = \max_{1\leq i' \leq H', 1 \leq j' \leq W'} x_{i''+i-1',j''+j'-1,d}.
\]
resulting in an output of size $\by\in\real^{H''\times W'' \times D}$, similar to the convolution operator of Section~\ref{s:convolution}. Sum-pooling computes the average of the values instead:
\[
y_{i''j''d} = \frac{1}{W'H'}
\sum_{1\leq i' \leq H', 1 \leq j' \leq W'} x_{i''+i'-1,j''+j'-1,d}.
\]
Detailed calculation of the derivatives are provided in Section~\ref{s:impl-pooling}.

\paragraph{Padding and stride.} Similar to the convolution operator of Sect.~\ref{s:convolution}, \verb!vl_nnpool! supports padding the input; however, the effect is different from padding in the convolutional block as pooling regions straddling the image boundaries are cropped. For max pooling, this is equivalent to extending the input data with $-\infty$; for sum pooling, this is similar to padding with zeros, but the normalization factor at the boundaries is smaller to account for the smaller integration area.

% ------------------------------------------------------------------
\section{Activation functions}\label{s:activation}
% ------------------------------------------------------------------

\vlnn supports the following activation functions:
%
\begin{itemize}
\item \emph{ReLU.} \verb!vl_nnrelu! computes the \emph{Rectified Linear Unit} (ReLU):
\[
 y_{ijd} = \max\{0, x_{ijd}\}.
\]

\item \emph{Sigmoid.} \verb!vl_nnsigmoid! computes the \emph{sigmoid}:
\[
 y_{ijd} = \sigma(x_{ijd}) = \frac{1}{1+e^{-x_{ijd}}}.
\]
\end{itemize}
%
See Section~\ref{s:impl-activation} for implementation details.

% ------------------------------------------------------------------
\section{Normalization}\label{s:normalization}
% ------------------------------------------------------------------

% ------------------------------------------------------------------
\subsection{Cross-channel normalization}\label{s:ccnormalization}
% ------------------------------------------------------------------

\verb!vl_nnnormalize! implements a cross-channel normalization operator. Normalization applied independently at each spatial location and groups of channels to get:
\[
 y_{ijk} = x_{ijk} \left( \kappa + \alpha \sum_{t\in G(k)} x_{ijt}^2 \right)^{-\beta},
\]
where, for each output channel $k$, $G(k) \subset \{1, 2, \dots, D\}$ is a corresponding subset of input channels. Note that input $\bx$ and output $\by$ have the same dimensions. Note also that the operator is applied across feature channels in a convolutional manner at all spatial locations.

See Section~\ref{s:impl-ccnormalization} for implementation details.

% ------------------------------------------------------------------
\subsection{Batch normalization}\label{s:bnorm}
% ------------------------------------------------------------------

\verb!vl_nnbnorm! implements batch normalization~\cite{ioffe2015}. Batch normalization is somewhat different from other neural network blocks in that it performs computation across images/feature maps in a batch (whereas most blocks process different images/feature maps individually). \verb!y = vl_nnbnorm(x, w, b)! normalizes each channel of the feature map $\mathbf{x}$ averaging over spatial locations and batch instances. Let $T$ the batch size; then
\[
\mathbf{x}, \mathbf{y} \in \mathbb{R}^{H \times W \times K \times T},
\qquad\mathbf{w} \in \mathbb{R}^{K},
\qquad\mathbf{b} \in \mathbb{R}^{K}.
\]
Note that in this case the input and output arrays are explicitly treated as 4D tensors in order to work with a batch of feature maps. The tensors  $\mathbf{w}$ and $\mathbf{b}$ define component-wise multiplicative and additive constants. The output feature map is given by
\[
y_{ijkt} = w_k \frac{x_{ijkt} - \mu_{k}}{\sqrt{\sigma_k^2 + \epsilon}} + b_k,
\quad
\mu_{k} = \frac{1}{HWT}\sum_{i=1}^H \sum_{j=1}^W \sum_{t=1}^{T} x_{ijkt},
\quad
\sigma^2_{k} = \frac{1}{HWT}\sum_{i=1}^H \sum_{j=1}^W \sum_{t=1}^{T} (x_{ijkt} - \mu_{k})^2.
\]
See Section~\ref{s:impl-bnorm} for implementation details.

% ------------------------------------------------------------------
\subsection{Spatial normalization}\label{s:spnorm}
% ------------------------------------------------------------------

\verb!vl_nnspnorm! implements spatial normalization. Spatial normalization operator acts on different feature channels independently and rescales each input feature by the energy of the features in a local neighborhood . First, the energy of the features is evaluated in a neighbourhood $W'\times H'$
\[
n_{i''j''d}^2 = \frac{1}{W'H'}
\sum_{1\leq i' \leq H', 1 \leq j' \leq W'} x^2_{
i''+i'-1-\lfloor \frac{H'-1}{2}\rfloor,
j''+j'-1-\lfloor \frac{W'-1}{2}\rfloor,
d}.
\]
In practice, the factor $1/W'H'$ is adjusted at the boundaries to account for the fact that neighbors must be cropped. Then this is used to normalize the input:
\[
y_{i''j''d} = \frac{1}{(1 + \alpha n_{i''j''d}^2)^\beta} x_{i''j''d}.
\]
See Section\ref{s:spnorm} for implementation details.

% ------------------------------------------------------------------
\subsection{Softmax}\label{s:softmax}
% ------------------------------------------------------------------

\verb!vl_nnsoftmax! computes the softmax operator:
\[
 y_{ijk} = \frac{e^{x_{ijk}}}{\sum_{t=1}^D e^{x_{ijt}}}.
\]
Note that the operator is applied across feature channels and in a convolutional manner at all spatial locations. Softmax can be seen as the combination of an activation function (exponential) and a normalization operator. See Section~\ref{s:impl-softmax} for implementation details.

% ------------------------------------------------------------------
\section{Losses and comparisons}\label{s:losses}
% ------------------------------------------------------------------

% ------------------------------------------------------------------
\subsection{Log-loss}\label{s:loss}
% ------------------------------------------------------------------

\verb!vl_logloss! computes the \emph{logarithmic loss}
\[
 y = \ell(\bx,c) = - \sum_{ij} \log x_{ijc}
\]
where $c \in \{1,2,\dots,D\}$ is the ground-truth class. Note that the operator is applied across input channels in a convolutional manner, summing the loss computed at each spatial location into a single scalar.  See Section~\ref{s:impl-loss} for implementation details.

% ------------------------------------------------------------------
\subsection{Softmax log-loss}\label{s:sfloss}
% ------------------------------------------------------------------

\verb!vl_softmaxloss! combines the softmax layer and the log-loss into one step for improved numerical stability. It computes
\[
y = - \sum_{ij} \left(
x_{ijc} - \log \sum_{d=1}^D e^{x_{ijd}}
\right)
\]
where $c$ is the ground-truth class. See Section~\ref{s:impl-sfloss} for implementation details.

% ------------------------------------------------------------------
\subsection{$p$-distance}\label{s:pdistance}
% ------------------------------------------------------------------

The \verb!vl_nnpdist! function computes the $p$-th power $p$-distance between the vectors in the input data $\bx$ and a target $\bar\bx$:
\[
  y_{ij} = \left(\sum_d |x_{ijd} - \bar x_{ijd}|^p\right)^\frac{1}{p}
\]
Note that this operator is applied convolutionally, i.e. at each spatial location $ij$ one extracts and compares vectors $x_{ij:}$. By specifying the option \verb!'noRoot', true! it is possible to compute a variant omitting the root:
\[
  y_{ij} = \sum_d |x_{ijd} - \bar x_{ijd}|^p, \qquad p > 0.
\]
See Section~\ref{s:impl-pdistance} for implementation details.

%% ------------------------------------------------------------------
%\subsection{Product}\label{s:product}
%% ------------------------------------------------------------------
%
%\[
% y_{ijd} = x^{(1)}_{ijd} x^{(2)}_{ijd}
%\]
%
%\paragraph{Implementation details.}
%\[
% \frac{dz}{dx^{(1)}_{ijd}}
%  =
% \sum_{i''j''d''}
%  \frac{dz}{dy_{i''j''d''}} 
%  \frac{dy_{i''j''d''}}{dx^{(1)}_{ijd}}
%  =
%  \frac{dz}{dy_{ijd''}} 
%  x^{(2)}_{ijd},
%  \qquad
%  \frac{dz}{dx^{(2)}_{ijd}}
%   =
%  \frac{dz}{dy_{ijd}} 
%  x^{(1)}_{ijd}.
%\]
%
%
%% ------------------------------------------------------------------
%\subsection{Split}\label{s:split}
%% ------------------------------------------------------------------
%
%\[
% y_{ijd}^{(1)} = x_{ijd}, \qquad y_{ijd}^{(2)} = x_{ijd}
%\]
%
%\[
% \frac{dz}{dx_{ijd}} =
%\sum_{i''j''d''} 
% \frac{dz}{dy_{i''j''d''}^{(1)}}
%  \frac{dy_{i''j''d''}^{(1)}}{dx_{ijd}}
% +
%  \frac{dz}{dy_{i''j''d''}^{(2)}}
%  \frac{dy_{i''j''d''}^{(2)}}{dx_{ijd}}
%\]

% ------------------------------------------------------------------
\chapter{Geometry}\label{s:geometry}
% ------------------------------------------------------------------

This chapter looks at the geometry of the CNN input-output mapping.

% ------------------------------------------------------------------
\section{Receptive fields and transformations}\label{s:receptive}
% ------------------------------------------------------------------

Very often it is interesting to map a convolutional operator back to image space. Due to various downsampling and padding operations, this is not entirely trivial and analysed in this section. Considering a 1D example will be enough; hence, consider an input signal $x_i$, a filter $f_{i'}$ and an output signal $y_{i''}$, where
\[
   1 - P_w^- \leq i \leq W + P_w^+,  \qquad 1 \leq i' \leq W'.
\]
where $P_w^-,P_w^+ \geq 0$ is the amount of padding applied to the input signal to the left and right respectively. The output $y_i$ is obtained by applying a filter of size $W'$ in the following range
\[
   - P_w^- + S (i'' - 1) + [1, W'] = [1 - P_w^- + S(i'' - 1), - P_w^- + S(i''-1) + W']
\]
where $S$ is the subsampling factor. For $i'' = 1$, the leftmost sample of this window falls at the leftmost sample of the padded input signal. The rightmost sample $W'' \geq i''$ of the output signal is obtained by guaranteeing that the filter window stays within the padded input signal:
\[
- P_w^- + S(i''-1) + W' \leq W + P_w^+
\qquad\Rightarrow\qquad
i'' \leq \frac{W - W' + P_w^- + P_w^+}{S} + 1
\]
Since $i''$ is an integer quantity, the maximum value is:
\begin{equation}\label{e:width}
W'' = 
\left\lfloor \frac{W - W' + P_w^- + P_w^+}{S} \right\rfloor
+ 1.
\end{equation}
Note that $W''$ is smaller than 1 if $W' > W + P_w^- + P_w^+$; in this case the filter $f_{i'}$ is wider than the padded input signal $x_i$ and the domain of $y_{i''}$ becomes empty (this generates an error in most MatConvNet functions).

Now consider a sequence of convolutional operators. One starts from an input signal $x_0$ of width $W_0$ and applies a sequence of  operators of parameters
$
(P_{w1}^-,  P_{w1}^+, S_1, W'_1), 
$
$
(P_{w2}^-,  P_{w2}^+, S_2, W'_2), 
$
$\dots,
$
$
(P_{wL}^-,  P_{wL},^+ S_L, W'_L)
$
to obtain signals $x_1,x_2,\dots,x_L$ of width $W_1,W_2,\dots,W_L$ respectively. First, note that the widths of these signals are obtained from $W_0$ and the operator parameters by a recursive application of~\eqref{e:width}. Due to the flooring operation, unfortunately it does not seem possible to simplify significantly the resulting expression. However, disregarding this operation one obtains the approximate expression
\[
W_l \approx
\frac{W_0}{\prod_{p=1}^l S_q}
-
\sum_{p=1}^l
\frac
{W'_p - P_{wp}^- - P_{wp}^+  - S_{p}}
{\prod_{q=p}^l S_q}.
\]
This expression is exact when $S_1 = S_2 =\dots = S_l =1$:
\[
W_l = W_0 - \sum_{p=1}^l (W'_p - P_{wp}^- - P_{wp}^+ - 1).
\]
Note in particular that without padding and filters $W_p' >1$ the widths decrease with depth. 


Next, we are interested in computing the samples $x_{0,i_0}$ of the input signal that affect a particular sample $x_{L,i_L}$ at the end of the chain (this we call the \emph{receptive field} of the operator). Suppose that at level $l$ we have determined that samples $i_l \in [I_l^-(i_L), I_l^+(i_L)]$ affect $x_{L,i_L}$ and compute the samples at the level below. These are given by the union of filter applications:
\[
 \cup_{I_l^-(i_L) \leq i_l \leq I_l^+(i_L)} \left(-P_{wl}^- + S_l (i_l-1)  + [1, W_l'] \right).
\]
Hence we find the recurrences:
\begin{align*}
 I_{l-1}^-(i_L) &= - P_{wl}^- + S_l (I_l^-(i_L)-1) + 1,
 \\
 I_{l-1}^+(i_L) &= - P_{wl}^- + S_l (I_l^+(i_L)-1) + W_l'.
\end{align*}
Given the base case $I_L^-(i_L) = I_L^+(i_L) = i_L$,  one gets:
\begin{align*}
I_{l}^-(i_L)
&=
1 
+ \left(\prod_{p=l+1}^L S_p\right) (i_L - 1)
- \sum_{p =l+1}^L  \left(\prod_{q=l+1}^{p-1} S_q\right)P_{wp}^-,
\\
I_{l}^-(i_L)
&=
1 
+ \left(\prod_{p=l+1}^L S_p\right) (i_L - 1)
+ \sum_{p = l+1}^L  \left(\prod_{q=l+1}^{p-1} S_q\right)(W_p' - 1 - P_{wp}^-).
\end{align*}
We can now compute several quantities of interest. First, the receptive field width on the image is:
\[
 \Delta 
 = I_0^+(i_L) - I_0^-(i_L) + 1
 = 1+ \sum_{p = 1}^L  \left(\prod_{q=1}^{p-1} S_q\right)(W_p' - 1).
\]
Second, the leftmost sample of the input signal $x_{0,i_0}$ affecting output sample $x_{l,i_L}$ is
\[
i_0(i_L)  =
1 
+ \left(\prod_{p=1}^L S_p\right) (i_L - 1)
- \sum_{p =1}^L  \left(\prod_{q=1}^{p-1} S_q\right)P_{wp}^-
\]
Note that the effect of padding at different layers accumulates, resulting in an overall padding potentially larger than the padding $P_{w1}^-$ specified by the first operator (i.e. $i_0(1) \leq - P_{w1}^-$).

Finally, it is interesting to reparametrise these coordinates as if the discrete signal were continuous and if (as it is usually the case) all the convolutional operators are ``centred''. For example, the operators could be filters with an odd size $W_l'$ equal to the delta function (e.g. for $W_l'=5$ then $f_l = (0,0,1,0,0)$). Let $u_L = i_L$ be the ``continuous'' version of index $i_L$, where each discrete sample corresponds, as per MATLAB's convention, to a tile of extent $u_L \in [-1/2,1/2] + i_L$ (hence $u_L$ is the centre coordinate of a sample). As before, $x_L(u_L)$ is obtained by applying an operator to the input signal $x_0$; the centre of the support of this operator falls at coordinate:
\[
  u_0(u_L) = \alpha\, (u_L - 1) + \beta = i_0(u_L) + \frac{\Delta-1}{2} = \alpha\, (u_L - 1) + \beta.
\]
Hence
\[
\alpha = \prod_{p=1}^L S_p,
\qquad
\beta
 = 
1
+ 
\sum_{p = 1}^L  \left(\prod_{q=1}^{p-1} S_q\right)
\left(
\frac{W_p' - 1}{2}  - P_{wp}^-
\right).
\]
Note in particular that the offset is zero if $S_1 =\dots=S_L=1$ and $P_{wp}^- = (W_p'-1)/2$ as in this case the padding is just enough such that the leftmost application of each convolutional operator has the centre that falls on the first sample of the corresponding input signal.
% ------------------------------------------------------------------
\chapter{Implementation details}\label{s:impl}
% ------------------------------------------------------------------

This chapter contains calculations and details.

% ------------------------------------------------------------------
\section{Convolution}\label{s:impl-convolution}
% ------------------------------------------------------------------

It is often convenient to express the convolution operation in matrix form. To this end, let $\phi(\bx)$ the \verb!im2row! operator, extracting all $W' \times H'$ patches from the map $\bx$ and storing them as rows of a $(H''W'') \times (H'W'D)$ matrix. Formally, this operator is given by:
\[
   [\phi(\bx)]_{pq} \underset{(i,j,d)=t(p,q)}{=} x_{ijd}
\]
where the index mapping $(i,j,d) = t(p,q)$ is
\[
 i = i''+i'-1, \quad
 j = j''+j'-1, \quad
 p = i'' + H'' (j''-1), \quad
 q = i' + H'(j'-1) + H'W' (d-1).
\]
It is also useful to define the ``transposed'' operator \verb!row2im!:
\[
   [\phi^*(M)]_{ijd}
   =
   \sum_{(p,q) \in t^{-1}(i,j,d)}
   M_{pq}.
\]
Note that $\phi$ and $\phi^*$ are linear operators. Both can be expressed by a matrix $H\in\real^{(H''W''H'W'D) \times(HWD)}$ such that
\[
  \vv(\phi(\bx)) = H \vv(\bx), \qquad 
  \vv(\phi^*(M)) = H^\top \vv(M).
\]
Hence we obtain the following expression for the vectorized output (see~\cite{kinghorn96integrals}):
\[
 \vv\by = 
 \vv\left(\phi(\bx) F\right)
 =
 \begin{cases}
 (I \otimes \phi(\bx)) \vv F, & \text{or, equivalently,} \\
 (F^\top \otimes I) \vv \phi(\bx),
 \end{cases}
\]
where $F\in\mathbb{R}^{(H'W'D)\times K}$ is the matrix obtained by reshaping the array $\bff$ and $I$ is an identity matrix of suitable dimensions. This allows obtaining the following formulas for the derivatives:
\[
\frac{dz}{d(\vv F)^\top}
=
\frac{dz}{d(\vv\by)^\top}
(I \otimes \phi(\bx))
= \vv\left[ 
\phi(\bx)^\top 
\frac{dz}{dY}
\right]^\top
\]
where $Y\in\real^{(H''W'')\times K}$ is the matrix obtained by reshaping the array $\by$. Likewise:
\[
\frac{dz}{d(\vv \bx)^\top}
=
\frac{dz}{d(\vv\by)^\top}
(F^\top \otimes I)
\frac{d\vv \phi(\bx)}{d(\vv \bx)^\top}
=
\vv\left[ 
\frac{dz}{dY}
F^\top
\right]^\top
H
\]
In summary, after reshaping these terms we obtain the formulas:
\[
\boxed{
\vv\by = 
 \vv\left(\phi(\bx) F\right),
\qquad
\frac{dz}{dF}
=
\phi(\bx)^\top\frac{d z}{d Y},
\qquad
\frac{d z}{d X}
=
\phi^*\left(
\frac{d z}{d Y}F^\top
\right)
}
\]
where $X\in\real^{(H'W')\times D}$ is the matrix obtained by reshaping $\bx$. Notably, these expressions are used to implement the convolutional operator; while this may seem inefficient, it is instead a fast approach when the number of filters is large and it allows leveraging fast BLAS and GPU BLAS implementations.

% ------------------------------------------------------------------
\section{Convolution transpose}\label{s:impl-convolution-transpose}
% ------------------------------------------------------------------

In order to understand the definition of convolution transpose, let $\by$ obtained from $\bx$ by the convolution operator as defined in Section~\ref{s:convolution} (including padding and downsampling).  Since this is a linear operation, it can be rewritten as $\vv \by = M \vv\bx$ for a suitable matrix $M$; convolution transpose computes instead $\vv \bx = M^\top \vv \by$.  While this is simple to describe in term of matrices, what happens in term of indexes is tricky. In order to derive a formula for the convolution transpose, start from standard convolution (for a 1D signal):
\[
   y_{i''} = \sum_{i'=1}^{H'} f_{i'} x_{S (i''-1) + i' - P_h^-}, 
   \quad
    1 \leq i'' \leq 1 + \left\lfloor \frac{H - H' + P_h^- + P_h^+}{S} \right\rfloor,
\]
where $S$ is the downsampling factor, $P_h^-$ and $P_h^+$ the padding, $H$ the length of the input signal, $\bx$ and $H'$ the length of the filter $\bff$. Due to padding, the index of the input data $\bx$ may exceed the range $[1,H]$; we implicitly assume that the signal is zero padded outside this range.

In order to derive an expression of the convolution transpose,  we make use of the identity $\vv \by^\top (M \vv \bx) = (\vv \by^\top M) \vv\bx = \vv\bx^\top (M^\top \vv\by)$. Expanding this in formulas:
\begin{align*}
\sum_{i''=1}^b y_{i''} 
\sum_{i'=1}^{W'} f_{i'} x_{S (i''-1) + i'  -P_h^-}
&=
\sum_{i''=-\infty}^{+\infty}
\sum_{i'=-\infty}^{+\infty} 
y_{i''}\ f_{i'}\ x_{S (i''-1) + i'  -P_h^-}
\\
&=
\sum_{i''=-\infty}^{+\infty}
\sum_{k=-\infty}^{+\infty} 
y_{i''}\ f_{k-S(i'' -1) + P_h^-}\ x_{k}
\\
&=
\sum_{i''=-\infty}^{+\infty}
\sum_{k=-\infty}^{+\infty} 
y_{i''}%
\ %
f_{%
(k-1+ P_h^-) \bmod S +
S \left(1 -i''  + \left\lfloor \frac{k-1+ P_h^-}{S} \right\rfloor\right)+1
}\ x_{k}
\\
&=
\sum_{k=-\infty}^{+\infty} 
x_{k}
\sum_{q=-\infty}^{+\infty}
y_{\left\lfloor \frac{k-1+ P_h^-}{S} \right\rfloor + 2 - q}
\ %
f_{(k-1+ P_h^-)\bmod S +S(q - 1)+1}.
\end{align*}
Summation ranges have been extended to infinity by assuming that all signals are zero padded as needed. In order to recover such ranges, note that $k \in [1,H]$ (since this is the range of elements of $\bx$ involved in the original convolution). Furthermore, $q\geq 1$ is the minimum value of $q$ for which the filter $\bff$ is non zero; likewise, $q\leq \lfloor (H'-1)/2\rfloor +1$ is a fairly tight upper bound on the maximum value (although, depending on $k$, there could be an element less). Hence
\begin{equation}\label{e:convt-step}
 x_k = 
 \sum_{q=1}^{1 + \lfloor \frac{H'-1}{S} \rfloor}
y_{\left\lfloor \frac{k-1+ P_h^-}{S} \right\rfloor + 2 - q}\ %
f_{(k-1+ P_h^-)\bmod S +S(q - 1)+1},
\qquad k=1,\dots, H.
\end{equation}
Here $H$, $H'$ and $H''$ are related by:
\[
H'' = 1+ \left\lfloor \frac{H - H' + P_h^- + P_h^+}{S} \right\rfloor.
\]
If $H''$ is now given as input, it is not possible to recover $H$ uniquely; instead, all the following values are possible
\[
   S_h (H''-1) +H' -  P_h^- - P_h^+ \leq H < S_h H'' +H' -  P_h^- - P_h^+.
\]
We use the tighter definition and set $H =  S_h (H''-1) +H' -  P_h^- - P_h^+$.
Note that the summation extrema in \eqref{e:convt-step} can be refined slightly to account for the finite size of $\by$ and $\bw$:
\begin{multline*}
\max\left\{
1, 
\left\lfloor \frac{k-1 + P_h^-}{S} \right\rfloor + 2 - H''
\right\}
\leq q \\
\leq
1 +\min\left\{
\left\lfloor \frac{H'-1-(k-1+ P_h^-)\bmod S}{S} \right\rfloor, 
\left\lfloor \frac{k-1 + P_h^-}{S} \right\rfloor
\right\}.
\end{multline*}

% ------------------------------------------------------------------
\section{Spatial pooling}\label{s:impl-pooling}
% ------------------------------------------------------------------

Since max pooling simply select for each output element an input element, the relation can be expressed in matrix form as
$
    \vv\by = S(\bx) \vv \bx
$
for a suitable selector matrix $S(\bx)\in\{0,1\}^{(H''W''D) \times (HWD)}$. The derivatives can the be written as:
$
\frac{d z}{d (\vv \bx)^\top}
=
\frac{d z}{d (\vv \by)^\top}
S(\bx),
$
for all but a null set of points, where the operator is not differentiable (this usually does not pose problems in optimization by stochastic gradient). For max-pooling, similar relations exists with two differences: $S$ does not depend on the input $\bx$ and it is not binary, in order to account for the normalization factors. In summary, we have the expressions:
\begin{equation}\label{e:max-mat}
\boxed{
\vv\by = S(\bx) \vv \bx,
\qquad
\frac{d z}{d \vv \bx}
=
S(\bx)^\top
\frac{d z}{d \vv \by}.
}
\end{equation}



% ------------------------------------------------------------------
\section{Activation functions}\label{s:impl-activation}
% ------------------------------------------------------------------

% ------------------------------------------------------------------
\subsection{ReLU}\label{s:impl-relu}
% ------------------------------------------------------------------

The ReLU operator can be expressed in matrix notation as
\[
\vv\by = \diag\bfs \vv \bx,
\qquad
\frac{d z}{d \vv \bx}
=
\diag\bfs
\frac{d z}{d \vv \by}
\]
where $\bfs = [\vv \bx > 0] \in\{0,1\}^{HWD}$ is an indicator vector.

% ------------------------------------------------------------------
\subsection{Sigmoid}\label{s:impl-sigmoid}
% ------------------------------------------------------------------

The derivative of the sigmoid function is given by
\begin{align*}
\frac{dz}{dx_{ijk}}
&= 
\frac{dz}{d y_{ijd}} 
\frac{d y_{ijd}}{d x_{ijd}}
=
\frac{dz}{d y_{ijd}} 
\frac{-1}{(1+e^{-x_{ijd}})^2} ( - e^{-x_{ijd}})
\\
&=
\frac{dz}{d y_{ijd}} 
y_{ijd} (1 - y_{ijd}).
\end{align*}
In matrix notation:
\[
\frac{dz}{d\bx} = \frac{dz}{d\by} \odot 
\by \odot 
(\mathbf{1}\mathbf{1}^\top - \by).
\]

% ------------------------------------------------------------------
\section{Normalization}\label{s:normalization}
% ------------------------------------------------------------------

% ------------------------------------------------------------------
\subsection{Cross-channel normalization}\label{s:impl-ccnormalization}
% ------------------------------------------------------------------

The derivative is easily computed as:
\[
\frac{dz}{d x_{ijd}}
=
\frac{dz}{d y_{ijd}}
L(i,j,d|\bx)^{-\beta}
-2\alpha\beta x_{ijd}
\sum_{k:d\in G(k)}
\frac{dz}{d y_{ijk}}
L(i,j,k|\bx)^{-\beta-1} x_{ijk} 
\]
where
\[
 L(i,j,k|\bx) = \kappa + \alpha \sum_{t\in G(k)} x_{ijt}^2.
\]

% ------------------------------------------------------------------
\subsection{Batch normalization}\label{s:impl-bnorm}
% ------------------------------------------------------------------

The derivative of the input with respect to the network output is computed as follows:
\[
\frac{dz}{dx_{ijkt}} = \sum_{i''j''k''t''}
\frac{dz}{d y_{i''j''k''t''}} 
\frac{d y_{i''j''k''t''}}{d x_{ijkt}}.
\]
Since feature channels are processed independently, all terms with $k''\not=k$ are null. Hence
\[
\frac{dz}{dx_{ijkt}} = \sum_{i''j''t''}
\frac{dz}{d y_{i''j''kt''}} 
\frac{d y_{i''j''kt''}}{d x_{ijkt}},
\]
where
\[
\frac{d y_{i''j''kt''}}{d x_{ijkt}} 
=
w_k
\left(\delta_{i=i'',j=j'',t=t''} - \frac{d \mu_k}{d x_{ijkt}}\right)
\frac{1}{\sqrt{\sigma^2_k + \epsilon}}
-
\frac{w_k}{2}
\left(x_{i''j''kt''} - \mu_k\right)
\left(\sigma_k^2 + \epsilon \right)^{-\frac{3}{2}}
\frac{d \sigma_k^2}{d x_{ijkt}},
\]
the derivatives with respect to the mean and variance are computed as follows:
\begin{align*}
\frac{d \mu_k}{d x_{ijkt}} &= \frac{1}{HWT},
\\
\frac{d \sigma_k^2}{d x_{i'j'kt'}}
&=
\frac{2}{HWT}
\sum_{ijt}
\left(x_{ijkt} - \mu_k \right)
\left(\delta_{i=i',j=j',t=t'} - \frac{1}{HWT} \right)
=
\frac{2}{HWT} \left(x_{i'j'kt'} - \mu_k \right),
\end{align*}
and $\delta_E$ is the indicator function of the event $E$. Hence
\begin{align*}
\frac{dz}{dx_{ijkt}}
&=
\frac{w_k}{\sqrt{\sigma^2_k + \epsilon}}
\left(
\frac{dz}{d y_{ijkt}} 
-
\frac{1}{HWT}\sum_{i''j''kt''}
\frac{dz}{d y_{i''j''kt''}} 
\right)
\\
&-
\frac{w_k}{2(\sigma^2_k + \epsilon)^{\frac{3}{2}}}
\sum_{i''j''kt''}
\frac{dz}{d y_{i''j''kt''}} 
\left(x_{i''j''kt''} - \mu_k\right)
\frac{2}{HWT} \left(x_{ijkt} - \mu_k \right)
\end{align*}
i.e.
\begin{align*}
\frac{dz}{dx_{ijkt}}
&=
\frac{w_k}{\sqrt{\sigma^2_k + \epsilon}}
\left(
\frac{dz}{d y_{ijkt}} 
-
\frac{1}{HWT}\sum_{i''j''kt''}
\frac{dz}{d y_{i''j''kt''}} 
\right)
\\
&-
\frac{w_k}{\sigma^2_k + \epsilon}
\,
\frac{x_{ijkt} - \mu_k}{\sqrt{\sigma^2_k + \epsilon}}
\,
\frac{1}{HWT}
\sum_{i''j''kt''}
\frac{dz}{d y_{i''j''kt''}} 
\left(x_{i''j''kt''} - \mu_k\right).
\end{align*}

% ------------------------------------------------------------------
\subsection{Spatial normalization}\label{s:impl-spnorm}
% ------------------------------------------------------------------

The neighborhood norm $n^2_{i''j''d}$ can be computed by applying average pooling to $x_{ijd}^2$ using \verb!vl_nnpool! with a $W'\times H'$ pooling region, top padding $\lfloor \frac{H'-1}{2}\rfloor$, bottom padding $H'-\lfloor \frac{H-1}{2}\rfloor-1$, and similarly for the horizontal padding.

The derivative of spatial normalization can be obtained as follows:
\begin{align*}
\frac{dz}{dx_{ijd}} 
&= \sum_{i''j''d}
\frac{dz}{d y_{i''j''d}} 
\frac{d y_{i''j''d}}{d x_{ijd}}
\\
&=
\sum_{i''j''d}
\frac{dz}{d y_{i''j''d}} 
(1 + \alpha n_{i''j''d}^2)^{-\beta}
\frac{dx_{i''j''d}}{d x_{ijd}} 
-\alpha\beta
\frac{dz}{d y_{i''j''d}} 
(1 + \alpha n_{i''j''d}^2)^{-\beta-1}
x_{i''j''d}
\frac{dn_{i''j''d}^2}{d (x^2_{ijd})} 
\frac{dx^2_{ijd}}{d x_{ijd}}
\\
&=
\frac{dz}{d y_{ijd}} 
(1 + \alpha n_{ijd}^2)^{-\beta}
-2\alpha\beta x_{ijd}
\left[
\sum_{i''j''d}
\frac{dz}{d y_{i''j''d}} 
(1 + \alpha n_{i''j''d}^2)^{-\beta-1}
x_{i''j''d}
\frac{dn_{i''j''d}^2}{d (x_{ijd}^2)}
\right]
\\
&=
\frac{dz}{d y_{ijd}} 
(1 + \alpha n_{ijd}^2)^{-\beta}
-2\alpha\beta x_{ijd}
\left[
\sum_{i''j''d}
\eta_{i''j''d}
\frac{dn_{i''j''d}^2}{d (x_{ijd}^2)}
\right],
\quad
\eta_{i''j''d}=
\frac{dz}{d y_{i''j''d}} 
(1 + \alpha n_{i''j''d}^2)^{-\beta-1}
x_{i''j''d}
\end{align*}
Note that the summation can be computed as the derivative of the
\verb!vl_nnpool! block.

% ------------------------------------------------------------------
\subsection{Softmax}\label{s:impl-softmax}
% ------------------------------------------------------------------

Care must be taken in evaluating the exponential in order to avoid underflow or overflow. The simplest way to do so is to divide from numerator and denominator by the maximum value:
\[
 y_{ijk} = \frac{e^{x_{ijk} - \max_d x_{ijd}}}{\sum_{t=1}^D e^{x_{ijt}- \max_d x_{ijd}}}.
\]
The derivative is given by:
\[
\frac{dz}{d x_{ijd}}
=
\sum_{k}
\frac{dz}{d y_{ijk}}
\left(
e^{x_{ijd}} L(\bx)^{-1} \delta_{\{k=d\}}
-
e^{x_{ijd}}
e^{x_{ijk}} L(\bx)^{-2}
\right),
\quad
L(\bx) = \sum_{t=1}^D e^{x_{ijt}}.
\]
Simplifying:
\[
\frac{dz}{d x_{ijd}}
=
y_{ijd} 
\left(
\frac{dz}{d y_{ijd}}
-
\sum_{k=1}^K
\frac{dz}{d y_{ijk}} y_{ijk}.
\right).
\]
In matrix for:
\[
  \frac{dz}{dX} = Y \odot \left(\frac{dz}{dY} 
  - \left(\frac{dz}{dY} \odot Y\right) \bone\bone^\top\right)
\]
where $X,Y\in\real^{HW\times D}$ are the matrices obtained by reshaping the arrays
$\bx$ and $\by$. Note that the numerical implementation of this expression is straightforward once the output $Y$ has been computed with the caveats above.

% ------------------------------------------------------------------
\section{Losses and comparisons}\label{s:impl-losses}
% ------------------------------------------------------------------

% ------------------------------------------------------------------
\subsection{Log-loss}\label{s:impl-loss}
% ------------------------------------------------------------------

The derivative is
\[
\frac{dz}{dx_{ijd}} = - \frac{dz}{dy} \frac{1}{x_{ijc}} \delta_{\{d = c\}}.
\]

% ------------------------------------------------------------------
\subsection{Softmax log-loss}\label{s:impl-sfloss}
% ------------------------------------------------------------------

The derivative is given by
\[
\frac{dz}{dx_{ijd}} 
= - \frac{dz}{dy} \left(\delta_{d=c} - y_{ijc}\right)
\]
where $y_{ijc}$ is the output of the softmax layer. In matrix form:
\[
\frac{dz}{dX} 
= - \frac{dz}{dy} \left(\bone^\top \bfe_c - Y\right)
\]
where $X,Y\in\real^{HW\times D}$ are the matrices obtained by reshaping the arrays
$\bx$ and $\by$ and $\bfe_c$ is the indicator vector of class $c$.

% ------------------------------------------------------------------
\subsection{$p$-distance}\label{s:impl-pdistance}
% ------------------------------------------------------------------

The derivative of the operator without root is given by:
\begin{align*}
\frac{dz}{dx_{ijd}}
&=
\frac{dz}{dy_{ij}}
p |x_{ijd} - \bar x_{ijd}|^{p-1} \operatorname{sign} (x_{ijd} - \bar x_{ijd}).
\end{align*}
The derivative of the operator with root is given by:
\begin{align*}
\frac{dz}{dx_{ijd}}
&=
\frac{dz}{dy_{ij}}
\frac{1}{p}
\left(\sum_{d'} |x_{ijd'} - \bar x_{ijd'}|^p \right)^{\frac{1}{p}-1}
p |x_{ijd} - \bar x_{ijd}|^{p-1} \sign(x_{ijd} - \bar x_{ijd})
\\
&= 
\frac{dz}{dy_{ij}}
\frac{|x_{ijd} - \bar x_{ijd}|^{p-1} \sign(x_{ijd} - \bar x_{ijd})}{y_{ij}^{p-1}}.
\end{align*}
The formulas simplify a little for $p=1,2$ which are therefore implemented as special cases.


\clearpage

% ------------------------------------------------------------------
\bibliographystyle{plain}
\bibliography{references,/Users/vedaldi/src/bibliography/vedaldi}
\end{document}
% ------------------------------------------------------------------



